\usepackage[brazilian]{babel} % Regras tipográficas
\usepackage{graphicx} % Required for inserting images
\usepackage[hmarginratio=1:1,top=32mm,columnsep=21pt]{geometry} % Margens do documento
\usepackage{amsmath}
\usepackage{amssymb}
\usepackage{float}
\usepackage{multicol}
\usepackage{hyperref}
\usepackage{lipsum}
\usepackage{caption}
\usepackage{subcaption}
\usepackage[dvipsnames]{xcolor} % Para a definição de cores
\usepackage[
	cachedir = \detokenize{.cache/minted}
]{minted}   % Para a inclusão literal de arquivos com sintaxe

\usepackage{siunitx} % Facilita o uso das unidades do SI
        \sisetup{output-decimal-marker = {,}} % Configura a vírgula como o separador decimal
        \sisetup{range-phrase = \text{--}}

\usepackage[hmarginratio=1:1,top=32mm,columnsep=21pt]{geometry} % Margens do documento
% \usepackage[hang, small,labelfont=bf,up,textfont=it,up]{caption} % Legendas customizadas pra tabelas e imagens
\usepackage{booktabs} % Tabelas variáveis

\usepackage[sc]{mathpazo} % Usando uma fonte diferente para o documento
        \usepackage[T1]{fontenc} % Use 8-bit encoding that has 256 glyphs
        \linespread{1.05} % Aumentando o espaçamento entre as linhas (a fonte não fica tão legal com o espaçamento padrão)
        \usepackage{microtype} % Não lembro o que isso faz

\usepackage{enumitem} % Listas customizadas
        \setlist[itemize]{noitemsep} % Pra tornar as listas mais compactas

\usepackage{abstract}
        \renewcommand{\abstractnamefont}{\normalfont\bfseries} % Deixa o "Resumo" em negrito
        \renewcommand{\abstracttextfont}{\normalfont\small\itshape} % Deixa o conteúdo do resumo em itálico

\usepackage{titlesec} % Customização do título
        \renewcommand\thesection{\Roman{section}} % Números romanos para as secções
        \renewcommand\thesubsection{\roman{subsection}} % Para as subsecções também
        \titleformat{\section}[block]{\large\scshape}{\thesection.}{1em}{} % Muda a aparência do título das secções
        \titleformat{\subsection}[block]{\large}{\thesubsection.}{1em}{} % Muda a aparência do título das subsecções

\usepackage{fancyhdr} % Cabeçalho
        \pagestyle{fancy} % Cabeçalho em todas as páginas
        \fancyhead{}
        \fancyfoot{}
        \fancyhead[C]{Implementação do QuickSort com Variação na Escolha do
	Pivô}% $\bullet$ Lucca Pellegrini $\bullet$ \today} % Custom header text
        \fancyfoot[C]{\thepage} % Custom footer text

\usepackage{titling} % Customização do título

\usepackage{url} % Pra ajudar a lidar com urls chatos

\usepackage{amsmath, amsthm, amssymb, amsfonts} % AMS-TeX pra equações (em geral) mais bonitas e pra umas outras coisas
\usepackage{csquotes}
\usepackage{svg}

%==============================================================================

\newcommand{\mintsrc}[3]{
	\inputminted[
		linenos = true,
		fontsize = \footnotesize,
		frame = lines,
		framesep=2em,
		rulecolor=\color{Gray},
		label=\fbox{\color{Black}#3},
		labelposition=topline
	]{#1}{../#2/#3}
}