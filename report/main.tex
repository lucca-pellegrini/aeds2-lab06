% LTeX: language=pt-BR

\documentclass[10pt,oneside,onecolumn]{article}
\usepackage[brazilian]{babel} % Regras tipográficas
\usepackage{graphicx} % Required for inserting images
\usepackage[hmarginratio=1:1,top=32mm,columnsep=21pt]{geometry} % Margens do documento
\usepackage{siunitx}
\usepackage{amsmath}
\usepackage{amssymb}
\usepackage{float}
\usepackage{multicol}
\usepackage{hyperref}
\usepackage{lipsum}
\usepackage{caption}
\usepackage{subcaption}
\usepackage{luacode}
\usepackage[dvipsnames]{xcolor} % Para a definição de cores
\usepackage[
	cachedir = \detokenize{.cache/minted}
]{minted}   % Para a inclusão literal de arquivos com sintaxe

\usepackage{siunitx} % Facilita o uso das unidades do SI
        \sisetup{output-decimal-marker = {,}} % Configura a vírgula como o separador decimal
        \sisetup{range-phrase = \text{--}}

\usepackage[hmarginratio=1:1,top=32mm,columnsep=21pt]{geometry} % Margens do documento
% \usepackage[hang, small,labelfont=bf,up,textfont=it,up]{caption} % Legendas customizadas pra tabelas e imagens
\usepackage{booktabs} % Tabelas variáveis

\usepackage[sc]{mathpazo} % Usando uma fonte diferente para o documento
        \usepackage[T1]{fontenc} % Use 8-bit encoding that has 256 glyphs
        \linespread{1.05} % Aumentando o espaçamento entre as linhas (a fonte não fica tão legal com o espaçamento padrão)
        \usepackage{microtype} % Não lembro o que isso faz

\usepackage{enumitem} % Listas customizadas
        \setlist[itemize]{noitemsep} % Pra tornar as listas mais compactas

\usepackage{abstract}
        \renewcommand{\abstractnamefont}{\normalfont\bfseries} % Deixa o "Resumo" em negrito
        \renewcommand{\abstracttextfont}{\normalfont\small\itshape} % Deixa o conteúdo do resumo em itálico

\usepackage{titlesec} % Customização do título
        \renewcommand\thesection{\Roman{section}} % Números romanos para as secções
        \renewcommand\thesubsection{\roman{subsection}} % Para as subsecções também
        \titleformat{\section}[block]{\large\scshape\centering}{\thesection.}{1em}{} % Muda a aparência do título das secções
        \titleformat{\subsection}[block]{\large}{\thesubsection.}{1em}{} % Muda a aparência do título das subsecções

\usepackage{fancyhdr} % Cabeçalho
        \pagestyle{fancy} % Cabeçalho em todas as páginas
        \fancyhead{}
        \fancyfoot{}
        \fancyhead[C]{Implementação do QuickSort com Variação na Escolha do
	Pivô}% $\bullet$ Lucca Pellegrini $\bullet$ \today} % Custom header text
        \fancyfoot[C]{\thepage} % Custom footer text

\usepackage{titling} % Customização do título

\usepackage{url} % Pra ajudar a lidar com urls chatos

\usepackage{amsmath, amsthm, amssymb, amsfonts} % AMS-TeX pra equações (em geral) mais bonitas e pra umas outras coisas
\usepackage{csquotes}
\usepackage{svg}

%==============================================================================

\newcommand{\mintsrc}[3]{
	\inputminted[
		linenos = true,
		fontsize = \footnotesize,
		frame = lines,
		framesep=2em,
		rulecolor=\color{Gray},
		label=\fbox{\color{Black}#3},
		labelposition=topline
	]{#1}{../#2/#3}
}

%==============================================================================

% \setlength{\droptitle}{-4\baselineskip} % Sobe o título um pouco pra economizar espaço

% \pretitle{\begin{center}\Huge\bfseries} % Formatação do título
% \posttitle{\end{center}} % Fecha a formatação

\title{{\huge\bfseries Quicksort e seu Pivô}\\
Laboratório 06}
\author{ % Autores
	\textsc{Lucca Pellegrini} \\
	\normalsize{842986}
	% \and
	% \textsc{Thomaz, João Pedro} \\
	% \normalsize{...}
	% \and
	% \textsc{Luiza Fonseca} \\
	% \normalsize{...}
}

\date{\today}

% \renewcommand{\maketitlehookd}{ % O resumo
% 	\begin{abstract}
% 	\noindent
% 	Nesse trabalho, implementamos o algoritmo \textit{quicksort} variando a
% 	escolha do pivô, e observando o desempenho de cada abordagem em quatro
% 	tipos de arranjos de inteiros: ordenados, ordenados reversos,
% 	parcialmente ordenados, e aleatórios --- de 1 a 10.000 elementos.
% 	\end{abstract}
% }

\begin{document}

%==============================================================================

\maketitle
\section{Introdução}

Nesse trabalho, implementamos o algoritmo \textit{quicksort} em \textit{C,}
variando a escolha do pivô, e observando o desempenho de cada abordagem em
quatro tipos de arranjos de inteiros: ordenados, ordenados reversos,
parcialmente ordenados, e aleatórios, de um a 10.000 elementos. Os vetores
parcialmente ordenados são gerados ao iterar sobre cada dez índices,
aleatoriamente, de um vetor ordenado, e o trocar com outro índice aleatório.
Para ordená-los, usamos o quicksort com quatro mecanismos para a escolha do
pivô: primeiro elemento, último elemento, elemento aleatório, e mediana de três
--- em que os elementos primeiro, do meio, e último são comparados e ordenados
entre si, e o mediano se torna o pivô.

A fim de facilitar a execução dos testes, o código foi modularizado em três
unidades principais: um arquivo \texttt{quicksort.c} contém a implementação do
algoritmo com cada estratégia de escolha do pivô, um arquivo \texttt{cron.c}
implementa um mecanismo para medir tempos de execução usando duas funções:
\mint{c}|void cron_start(void); unsigned long cron_stop(void);| em que a
primeira inicia um cronômetro, e a segunda retorna o tempo em nanosegundos
desde o seu início. Já o arquivo \texttt{main.c} recebe o número de iterações
por meio dos parâmetros do programa, executa o algoritmo para cada estratégia,
para cada tipo de vetor, e exibe a saída em formato \textit{csv}. Essa saída é
processada em Python usando a biblioteca \textit{Pandas}, e os gráficos são
gerados automáticamente com o \textit{matplotlib}. O programa completo se
encontra no Apêndice~\ref{apx:codigo}, e o restante do código está no
GitHub.\footnote{\url{https://github.com/lucca-pellegrini/aeds2-lab06}}

\newpage
\section{Resultados}
\subsection{Dados}

\input{../build/fig/tables.tex}

\begin{figure}[H]
	\centering
	\includegraphics[width=1\linewidth]{../build/fig/Ordenados_all_data.pdf}
	\captionsetup{skip=0pt} % Adjust the skip length here (default is 10pt)
	\caption{Tempos de execução para cada estratégia em vetores já ordenados.
	Isso é, vetores de forma $\begin{bmatrix} 1 & 2 & 3 & \cdots & n \end{bmatrix}$.}
	\label{fig:ord}
\end{figure}

\begin{figure}[H]
	\centering
	\includegraphics[width=1\linewidth]{../build/fig/Ordenados Reversos_all_data.pdf}
	\captionsetup{skip=0pt} % Adjust the skip length here (default is 10pt)
	\caption{Tempos de execução para cada estratégia em vetores ordenados de
	forma decrescente (ordem reversa). Isso é, vetores de forma
	$\begin{bmatrix} n & n-1 & n-2 & \cdots & 1 \end{bmatrix}$.}
	\label{fig:rev}
\end{figure}

\begin{figure}[H]
	\centering
	\includegraphics[width=1\linewidth]{../build/fig/Aleatórios_all_data.pdf}
	\captionsetup{skip=0pt} % Adjust the skip length here (default is 10pt)
	\caption{Tempos de execução para cada estratégia em vetores gerados
	aleatoriamente, em que cada elemento é um número independente entre 1 e o
	tamanho do vetor.}
	\label{fig:rand}
\end{figure}

\begin{figure}[H]
	\centering
	\includegraphics[width=1\linewidth]{../build/fig/Parcialmente Ordenados_all_data.pdf}
	\captionsetup{skip=0pt} % Adjust the skip length here (default is 10pt)
	\caption{Tempos de execução para cada estratégia em vetores parcialmente
	ordenados, gerados a partir de um vetor já ordenado, em que se seleciona
	10\% dos elementos, aleatoriamente, e os troca com outro elemento escolhido
	aleatoriamente.}
	\label{fig:part}
\end{figure}

\subsection{Discussão}

Destes resultados, o que mais se destaca --- especialmente nas figuras
\ref{fig:ord} e \ref{fig:rev} --- é que a escolha do pivô como o primeiro ou o
último elemento do vetor tem um impacto dramático no tempo de execução quando o
quicksort é aplicado a um vetor já ordenado ou ordenado de forma decrescente, a
ponto que, no caso do vetor com 10.000 elementos (veja, por exemplo, a
tabela~\ref{tbl:Ordenados}), há um aumento de quase 6000\% no tempo de execução
em relação ao pivô escolhido aleatoriamente --- mais de uma ordem de magnitude.
No caso desses vetores, é indiscutível que a estratégia mais eficaz foi a
mediana de três, provavelmente porque, selecionando o meio do vetor como pivô,
assim dividindo-o em dois nas chamadas recursivas, o número de chamadas se
aproxima ao máximo de $\Theta(\log_2 n)$, enquanto, no caso do primeiro ou
último elemento como pivô, por particionar o vetor de forma extremamente
desigual, sistematicamente escolhe o pior pivô, e se aproxima de $\Theta(n)$, o
pior caso possível do quicksort.

No caso dos vetores aleatórios (figura~\ref{fig:rand},
tabela~\ref{tbl:Aleatórios}), não se observa uma grande diferença nos tempos de
execução, já que não há correlação entre a posição dos elementos no vetor e a
escolha do pivô. Em um vetor aleatório, cada elemento tem a mesma probabilidade
de ser o pivô ideal, o que resulta em partições mais equilibradas na média. O
surpreendente é que a estratégia de escolher o primeiro ou o último elemento
tem desempenho melhor que os demais. Isso provavelmente se deve ao fato de que,
como um vetor de 10.000 números inteiros é relativamente pequeno, há um custo
perceptível em computar um número aleatório, ou em ordenar três elementos entre
si, procedimentos necessários para escolher um pivô aleatório, ou baseado em
mediana de três. Pode-se imaginar que, à medida que $n$ cresce, a diferença
entre os desempenhos de cada estratégia, nesse caso, se tornará cada vez menos
pronunciado.

Finalmente, no caso dos vetores parcialmente ordenados (figura~\ref{fig:part},
tabela~\ref{tbl:Parcialmente Ordenados}), nota-se novamente que a mediana de
três é a estratégia mais vantajosa, já que particiona o vetor de maneira
equilibrada. O que se destaca aqui é que, diferentemente do caso anterior,
apesar da estratégia de pivô aleatório ter o pior desempenho, isso não pode ser
explicado pelo custo de computar um pivô não trivial, uma vez que --- se fosse
o caso --- a mediana de três não teria o melhor desempenho. Na verdade, o que
provavelmente justifica a ineficiência do pivô aleatório nesse caso é o fato de
que, por ser completamente aleatório, não tem a oportunidade de aproveitar a
estrutura do vetor parcialmente ordenado, enquanto a mediana de três considera
três pontos distintos para fazer um particionamento equilibrado. Já a escolha
do primeiro ou último elemento, por outro lado, não é ineficiente pois, nesse
tipo de vetor, ainda há a probabilidade dos elementos nas extremidades estarem
próximos do valor mediano. Isso é, diferentemente do caso dos vetores ordenados
ou reversos, não há uma escolha sistemática do pior elemento.

%==============================================================================

\newpage
\appendix
\section{Programa Completo}\label{apx:codigo}

\mintsrc{c}{src}{main.c}
\newpage \mintsrc{c}{include}{quicksort.h}
\mintsrc{c}{src}{quicksort.c}
\newpage \mintsrc{c}{include}{vec.h}
\mintsrc{c}{src}{vec.c}
\newpage \mintsrc{c}{include}{cron.h}
\mintsrc{c}{src}{cron.c}
\newpage \mintsrc{c}{include}{util.h}
\mintsrc{c}{src}{util.c}

%==============================================================================
\end{document}