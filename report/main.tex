% LTeX: language=pt-BR

\documentclass[10pt,oneside,onecolumn]{article}
\usepackage[brazilian]{babel} % Regras tipográficas
\usepackage{graphicx} % Required for inserting images
\usepackage[hmarginratio=1:1,top=32mm,columnsep=21pt]{geometry} % Margens do documento
\usepackage{siunitx}
\usepackage{amsmath}
\usepackage{amssymb}
\usepackage{float}
\usepackage{multicol}
\usepackage{hyperref}
\usepackage{lipsum}
\usepackage{caption}
\usepackage{subcaption}
\usepackage{luacode}
\usepackage[dvipsnames]{xcolor} % Para a definição de cores
\usepackage[
	cachedir = \detokenize{.cache/minted}
]{minted}   % Para a inclusão literal de arquivos com sintaxe

\usepackage{siunitx} % Facilita o uso das unidades do SI
        \sisetup{output-decimal-marker = {,}} % Configura a vírgula como o separador decimal
        \sisetup{range-phrase = \text{--}}

\usepackage[hmarginratio=1:1,top=32mm,columnsep=21pt]{geometry} % Margens do documento
% \usepackage[hang, small,labelfont=bf,up,textfont=it,up]{caption} % Legendas customizadas pra tabelas e imagens
\usepackage{booktabs} % Tabelas variáveis

\usepackage[sc]{mathpazo} % Usando uma fonte diferente para o documento
        \usepackage[T1]{fontenc} % Use 8-bit encoding that has 256 glyphs
        \linespread{1.05} % Aumentando o espaçamento entre as linhas (a fonte não fica tão legal com o espaçamento padrão)
        \usepackage{microtype} % Não lembro o que isso faz

\usepackage{enumitem} % Listas customizadas
        \setlist[itemize]{noitemsep} % Pra tornar as listas mais compactas

\usepackage{abstract}
        \renewcommand{\abstractnamefont}{\normalfont\bfseries} % Deixa o "Resumo" em negrito
        \renewcommand{\abstracttextfont}{\normalfont\small\itshape} % Deixa o conteúdo do resumo em itálico

\usepackage{titlesec} % Customização do título
        \renewcommand\thesection{\Roman{section}} % Números romanos para as secções
        \renewcommand\thesubsection{\roman{subsection}} % Para as subsecções também
        \titleformat{\section}[block]{\large\scshape\centering}{\thesection.}{1em}{} % Muda a aparência do título das secções
        \titleformat{\subsection}[block]{\large}{\thesubsection.}{1em}{} % Muda a aparência do título das subsecções

\usepackage{fancyhdr} % Cabeçalho
        \pagestyle{fancy} % Cabeçalho em todas as páginas
        \fancyhead{}
        \fancyfoot{}
        \fancyhead[C]{Implementação do QuickSort com Variação na Escolha do
	Pivô}% $\bullet$ Lucca Pellegrini $\bullet$ \today} % Custom header text
        \fancyfoot[C]{\thepage} % Custom footer text

\usepackage{titling} % Customização do título

\usepackage{url} % Pra ajudar a lidar com urls chatos

\usepackage{amsmath, amsthm, amssymb, amsfonts} % AMS-TeX pra equações (em geral) mais bonitas e pra umas outras coisas
\usepackage{csquotes}
\usepackage{svg}

%==============================================================================

\newcommand{\mintsrc}[3]{
	\inputminted[
		linenos = true,
		fontsize = \footnotesize,
		frame = lines,
		framesep=2em,
		rulecolor=\color{Gray},
		label=\fbox{\color{Black}#3},
		labelposition=topline
	]{#1}{../#2/#3}
}

%==============================================================================

% \setlength{\droptitle}{-4\baselineskip} % Sobe o título um pouco pra economizar espaço

% \pretitle{\begin{center}\Huge\bfseries} % Formatação do título
% \posttitle{\end{center}} % Fecha a formatação

\title{{\huge\bfseries Implementação do QuickSort com Variação na Escolha do
	Pivô}\\
Laboratório 06}
\author{ % Autores
	\textsc{Lucca Pellegrini} \\
	\normalsize{842986}
	% \and
	% \textsc{Thomaz, João Pedro} \\
	% \normalsize{...}
	% \and
	% \textsc{Luiza Fonseca} \\
	% \normalsize{...}
}

\date{\today}

% \renewcommand{\maketitlehookd}{ % O resumo
% 	\begin{abstract}
% 	\noindent
% 	Nesse trabalho, implementamos o algoritmo \textit{QuickSort} variando a
% 	escolha do pivô, e observando o desempenho de cada abordagem em quatro
% 	tipos de arranjos de inteiros: ordenados, ordenados reversos,
% 	parcialmente ordenados, e aleatórios --- de 1 a 10.000 elementos.
% 	\end{abstract}
% }

\begin{document}

%==============================================================================

\maketitle
\section{Introdução}

Nesse trabalho, implementamos o algoritmo \textit{QuickSort} em \textit{C,}
variando a escolha do pivô, e observando o desempenho de cada abordagem em
quatro tipos de arranjos de inteiros: ordenados, ordenados reversos,
parcialmente ordenados, e aleatórios, de um a 10.000 elementos. Os vetores
parcialmente ordenados são gerados ao iterar sobre cada dez índices,
aleatoriamente, de um vetor ordenado, e o trocar com outro índice aleatório.
Para ordená-los, usamos o QuickSort com quatro mecanismos para a escolha do
pivô: primeiro elemento, último elemento, elemento aleatório, e mediana de três
--- em que os elementos primeiro, do meio, e último são comparados e ordenados
entre si, e o mediano se torna o pivô.

A fim de facilitar a execução dos testes, o código foi modularizado em três
unidades principais: um arquivo \texttt{quicksort.c} contém a implementação do
algoritmo com cada estratégia de escolha do pivô, um arquivo \texttt{cron.c}
implementa um mecanismo para medir tempos de execução usando duas funções:
\mint{c}|void cron_start(void); unsigned long cron_stop(void);| em que a
primeira inicia um cronômetro, e a segunda retorna o tempo em nanosegundos
desde o seu início. Já o arquivo \texttt{main.c} recebe o número de iterações
por meio dos parâmetros do programa, executa o algoritmo para cada estratégia,
para cada tipo de vetor, e exibe a saída em formato \textit{csv}. Essa saída é
processada em Python usando a biblioteca \textit{Pandas}, e os gráficos são
gerados automáticamente com o \textit{matplotlib}. O programa completo se
encontra no Apêndice~\ref{apx:codigo}, e o restante do código está no
GitHub.\footnote{\url{https://github.com/lucca-pellegrini/aeds2-lab06}}

\newpage
\section{Resultados}

\input{../build/fig/tables.tex}

\begin{figure}[H]
	\centering
	\includegraphics[width=1\linewidth]{../build/fig/Ordenados_all_data.pdf}
	\captionsetup{skip=0pt} % Adjust the skip length here (default is 10pt)
	\caption{Tempos de execução para cada estratégia em vetores já ordenados.}
	\label{fig:ord}
\end{figure}

\begin{figure}[H]
	\centering
	\includegraphics[width=1\linewidth]{../build/fig/Ordenados Reversos_all_data.pdf}
	\captionsetup{skip=0pt} % Adjust the skip length here (default is 10pt)
	\caption{Tempos de execução para cada estratégia em vetores ordenados de
	forma decrescente (ordem reversa).}
	\label{fig:rev}
\end{figure}

\begin{figure}[H]
	\centering
	\includegraphics[width=1\linewidth]{../build/fig/Aleatórios_all_data.pdf}
	\captionsetup{skip=0pt} % Adjust the skip length here (default is 10pt)
	\caption{Tempos de execução para cada estratégia em vetores gerados
	aleatoriamente, em que cada elemento é um número independente entre 1 e o
	tamanho do vetor.}
	\label{fig:rand}
\end{figure}

\begin{figure}[H]
	\centering
	\includegraphics[width=1\linewidth]{../build/fig/Parcialmente Ordenados_all_data.pdf}
	\captionsetup{skip=0pt} % Adjust the skip length here (default is 10pt)
	\caption{Tempos de execução para cada estratégia em vetores parcialmente
	ordenados, gerados a partir de um vetor já ordenado, em que se seleciona
	10\% dos elementos, aleatoriamente, e os troca com outro elemento escolhido
	aleatoriamente.}
	\label{fig:part}
\end{figure}

%==============================================================================

\newpage
\appendix
\section{Programa Completo}\label{apx:codigo}

\mintsrc{c}{src}{main.c}
\newpage \mintsrc{c}{include}{quicksort.h}
\mintsrc{c}{src}{quicksort.c}
\newpage \mintsrc{c}{include}{vec.h}
\mintsrc{c}{src}{vec.c}
\newpage \mintsrc{c}{include}{cron.h}
\mintsrc{c}{src}{cron.c}
\newpage \mintsrc{c}{include}{util.h}
\mintsrc{c}{src}{util.c}

%==============================================================================
\end{document}